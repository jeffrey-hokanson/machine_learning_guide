\documentclass{tufte-handout}
\usepackage{jeff}


\title{A Machine Learning Guide for Mathematicians}
\begin{document}
\maketitle

\newthought{Big Data, Machine Learning, Data Science} ---
these are all names for the area at the intersection of statistics, 
mathematics, and computer science that has captured so much attention.
 

\section{Estimating a Probability Distribution}
Given a collection of samples $x_j$ drawn from $X$ ($x_j = X(\omega_j)$),
estimate the probability measure associated with $X$.


\begin{itemize}
	\item Distinct events (e.g., dice roll)
	\item Continuous value (e.g., $x_j \in \R^p$)
	\begin{itemize}
		\item Parameteric density estimation -- 
			we have a model for $X$ that depends on unknown parameters
		\item Nonparametric density estimation\footnote{
				Nonparametric does not mean there is no model, 
				it means the model depends directly on $x_j$, 
				not on a fixed number of parameters derived from $x_j$.} --
			we don't have a model
	\end{itemize}
\end{itemize}

The first choice we need to make is $x_j$ a distinct event (e.g., dice roll) or 
continuous value (e.g., a real number).
If it is a distinct event, simply enumerate the events 


\subsection{Distinct Events}

\end{document}
